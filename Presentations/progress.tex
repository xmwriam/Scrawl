\documentclass{beamer}
\usetheme{CambridgeUS}
\usecolortheme{beaver}
\usefonttheme{serif}
\usepackage{hyperref}

\usebackgroundtemplate%
{%
    \includegraphics[width=\paperwidth,height=\paperheight]{bg.png}%
}

\title{Scrawl}
\subtitle{A Freeform Shared Journal Application}
\author{Mariam Eqbal \and Sheetal Lodhi}
\date{\today}

\begin{document}

\begin{frame}
    \titlepage
\end{frame}

% ─────────────────────────────────────────────
\section{Objective}
\begin{frame}{Objective}
    \begin{enumerate}
        \item To build a real-time, freeform shared canvas for two people to communicate expressively.
        \item To learn and establish proficiency in a modern full-stack web development stack (React, Node.js, Socket.io, Supabase).
        \item To design an intimate, asynchronous digital experience that feels like passing a physical journal.
    \end{enumerate}
\end{frame}

% ─────────────────────────────────────────────
\section{What is Scrawl?}
\begin{frame}{What is Scrawl?}
    \begin{enumerate}
        \item A private, infinite shared canvas between exactly \textbf{two people}.
        \item Users can place \textbf{freehand drawings}, \textbf{text}, and \textbf{images} anywhere on the canvas.
        \item Content is composed privately as a \textbf{draft}, then sent all at once - like sealing and posting a letter.
        \item Once sent, all content is \textbf{locked forever} - nothing can be moved or deleted.
        \item The canvas grows \textbf{downward infinitely}, like an endless shared notebook.
    \end{enumerate}
\end{frame}

% ─────────────────────────────────────────────
\section{Tech Stack}
\begin{frame}{Tech Stack}
    \begin{enumerate}
        \item \textbf{Frontend:} React (Vite), react-konva, React Router
        \item \textbf{Backend:} Node.js, Express, Socket.io
        \item \textbf{Database:} Supabase (PostgreSQL) for element persistence and room management
        \item \textbf{Storage:} Supabase Storage for uploaded images
        \item \textbf{Authentication:} Custom JWT + Google OAuth via \texttt{@react-oauth/google}
        \item \textbf{Deployment:} Vercel (frontend), Railway (backend)
    \end{enumerate}
\end{frame}

% ─────────────────────────────────────────────
\section{Features}
\begin{frame}{Features Implemented}
    \begin{enumerate}
        \item \textbf{Infinite canvas} - scrolls downward, clamped at the top like a real journal
        \item \textbf{Freehand drawing} - smooth lines with color picker and brush size control
        \item \textbf{Click-to-place text} - type anywhere on the canvas in any color
        \item \textbf{Image upload} - photos placed directly onto the canvas
        \item \textbf{Draft \& Send mechanic} - compose privately, send all at once
        \item \textbf{Real-time sync} - Socket.io keeps both users in sync instantly
        \item \textbf{Unique room URLs} - each journal gets a shareable room code
        \item \textbf{Persistence} - full canvas survives server restarts via Supabase
        \item \textbf{Google OAuth login} - one-click sign in, no passwords needed
        \item \textbf{Recency opacity} - older elements fade subtly, newer ones are crisp
    \end{enumerate}
\end{frame}

% ─────────────────────────────────────────────
\section{Division of Work}
\begin{frame}{Division of Work}
    \begin{enumerate}
        \item \textbf{Mariam} - Canvas architecture, drawing engine (Konva), real-time sync (Socket.io), deployment setup, core UX decisions
        \item \textbf{Sheetal} - Authentication system (JWT + Google OAuth), room management (create/join with room codes), color picker, brush size controls, stickers, timestamp display
    \end{enumerate}
\end{frame}

% ─────────────────────────────────────────────
\section{Timeline}
\begin{frame}{Timeline}
    \begin{enumerate}
        \item \textbf{Phase 1} - Local canvas: drawing and text placement
        \item \textbf{Phase 2} - Real-time sync with Socket.io (two tabs stay in sync)
        \item \textbf{Phase 3} - Unique room URLs, 2-person room lock
        \item \textbf{Phase 4} - Supabase persistence (canvas survives restarts)
        \item \textbf{Phase 5} - Image upload via backend + Supabase Storage
        \item \textbf{Phase 6} - Draft \& Send mechanic, element locking, glow on click
        \item \textbf{Phase 7} - Google OAuth, room codes, polish (recency opacity, scroll clamp, smoother lines)
        \item \textbf{Phase 8} - Deployment (Vercel + Railway)
    \end{enumerate}
\end{frame}

% ─────────────────────────────────────────────
\section{Repository}
\begin{frame}{Repository}
    \href{https://github.com/xmwriam/Scrawl}{\beamergotobutton{GitHub Repository}}
\end{frame}

% ─────────────────────────────────────────────
\section{Learnings}
\begin{frame}{Explicit Learnings}
    \begin{enumerate}
        \item Building real-time applications with \textbf{WebSockets} (Socket.io)
        \item Canvas programming with \textbf{Konva.js} - coordinates, layers, hit detection
        \item Full-stack architecture - React frontend talking to a Node/Express backend
        \item Database design and persistence with \textbf{Supabase} (PostgreSQL)
        \item Authentication flows - JWT tokens, Google OAuth, protected routes
        \item File upload and cloud storage (Supabase Storage)
    \end{enumerate}
\end{frame}

\begin{frame}{Implicit Learnings}
    \begin{enumerate}
        \item Debugging real-time race conditions and browser focus issues
        \item Managing shared state across two live users simultaneously
        \item Thinking carefully about \textbf{product design} before writing code
        \item Dividing and integrating work across a two-person team using Git
        \item Handling environment variables and secrets safely across environments
        \item The value of incremental, phase-by-phase development
    \end{enumerate}
\end{frame}

% ─────────────────────────────────────────────
\section{Challenges}
\begin{frame}{Challenges}
    \begin{enumerate}
        \item \textbf{Real-time sync} - React Strict Mode double-mounting caused the room to appear full before a second user joined; fixed by moving socket creation inside \texttt{useEffect} with proper cleanup.
        \item \textbf{Text input focus} - Konva canvas stole focus from the floating HTML input; fixed with a \texttt{setTimeout} defer pattern.
        \item \textbf{Supabase Storage RLS} - Row Level Security policies blocked image uploads from the frontend; solved by routing uploads through the backend using the service role key.
        \item \textbf{Smooth drawing} - Raw mouse events produced jagged lines; fixed by minimum distance threshold (3px) before sampling a new point.
        \item \textbf{Team coordination} - Merging Sheetal's auth system into the existing canvas architecture required careful integration.
    \end{enumerate}
\end{frame}

% ─────────────────────────────────────────────
\section{What Next}
\begin{frame}{What next}
    \begin{enumerate}
        \item \textbf{UI fixes} - Make the UI prettier
        \item \textbf{Canvas limitations} - Add limitations to the canvas for better UX
        \item \textbf{Drawing options} - Increase styling options
        \item \textbf{Home page} - Create ahome page where the user can access all their journals in one place

    \end{enumerate}
\end{frame}

% ─────────────────────────────────────────────
\section{Future Scope}
\begin{frame}{Future Scope}
    \begin{enumerate}
        \item \textbf{Group journals} - lift the 2-person cap; the userId-based architecture already supports this
        \item \textbf{Audio notes} - drop a short voice recording onto the canvas as a playable waveform
        \item \textbf{Time-lapse replay} - replay the journal from the beginning, stroke by stroke
        \item \textbf{Reactions} - hover over an element and leave a small floating reaction
        \item \textbf{Mobile app} - React Native port for a natural touch and stylus experience
        \item \textbf{Export} - download the full journal as a PDF keepsake
        \item \textbf{End-to-end encryption} - encrypt canvas content so even the server cannot read it
    \end{enumerate}
\end{frame}

% ─────────────────────────────────────────────
\section{Thank You}
\begin{frame}{ }
    \begin{center}
        \LARGE
        Thank You
    \end{center}
\end{frame}

\end{document}